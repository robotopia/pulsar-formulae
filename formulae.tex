\documentclass[twocolumn]{article}

\usepackage{amsmath}
\usepackage{fullpage}
\usepackage{hyperref}

\newcommand{\phase}{\varphi}
\newcommand{\deriv}[2]{\frac{d{#1}}{d{#2}}}
\newcommand{\derivn}[3]{\frac{d^{#3}{#1}}{d{#2}^{#3}}}
\newcommand{\pd}[2]{\frac{\partial {#1}}{\partial {#2}}}

\begin{document}

\tableofcontents

\section{Magnetic dipole equations}

\subsection{Field lines}

The magnetic field, in coordinate-free notation:
\begin{equation}
    \vec{B} = \frac{1}{r^3}\left[3(\vec{\mu}\cdot\hat{r})\hat{r} - \vec{\mu}\right]
\end{equation}
A single field line with maximum extent $R$, in polar coordinates:
\begin{equation}
    r = R\sin^2\theta
\end{equation}
The $x$- and $y$-coordinates of the above:
\begin{align}
    x &= R\sin^3\theta \\
    y &= R\sin^2\theta\cos\theta
\end{align}
And finally, the field line in Cartesian coordinates, where the $y$ axis is identified with the magnetic axis:
\begin{equation}
    y = x^{2/3} \sqrt{R^{2/3} - x^{2/3}}
\end{equation}

\subsection{Derivatives}

\begin{equation}
    \begin{aligned}
        \deriv{x}{\theta} &= 3R\sin^2\theta\cos\theta \\
          &= 3x^{2/3} \sqrt{R^{2/3} - x^{2/3}} \\
          &= 3y
    \end{aligned}
\end{equation}

\begin{equation}
    \begin{aligned}
        \deriv{y}{\theta} &= R\sin\theta(3\cos^2\theta - 1) \\
                          &= x^{1/3}(2R^{2/3} - 3x^{2/3})
    \end{aligned}
\end{equation}
The tangent of a field line:
\begin{equation}
    \begin{aligned}
        \deriv{y}{x} &= \frac{3\cos^2\theta-1}{3\sin\theta\cos\theta} \\
                     &= \frac{2R^{2/3} - 3x^{2/3}}{3x^{1/3}\sqrt{R^{2/3} - x^{2/3}}}
    \end{aligned}
\end{equation}
The beam opening angle, $\Gamma$, is
\begin{equation}
    \tan\Gamma = \deriv{x}{y} = \frac{3\sin\theta\cos\theta}{3\cos^2\theta-1}
\end{equation}
Solving the above for $\theta$, we have (for $-\pi \le \Gamma \le \pi$)
\begin{equation}
    \cos(2\theta) = \frac13 \left(\cos\Gamma\sqrt{8+\cos^2\Gamma} - \sin^2\Gamma\right)
\end{equation}
The arc length of a field line, as a function of $\theta$:
\begin{equation}
    \begin{aligned}
        \deriv{s}{\theta} &= \sqrt{\left(\deriv{x}{\theta}\right)^2 + \left(\deriv{y}{\theta}\right)^2} \\
                          &= R\sin\theta \sqrt{3\cos^2\theta + 1}
    \end{aligned}
\end{equation}

\subsection{Footpoints}

Let $r_p$ be the pulsar's radius, and $\theta_p$ by the angle the footpoint makes with the magnetic axis.
\begin{align}
    r_p &= R\sin^2\theta_p \\
    \theta_p &= \sin^{-1}\sqrt{\frac{r_p}{R}}
\end{align}
The $x$- and $y$-coordinates of the footpoint, $(x_p,y_p)$:
\begin{align}
    x_p &= r_p\sqrt{\frac{r_p}{R}} \\
    y_p &= r_p \sqrt{1 - \frac{r_p}{R}}
\end{align}

\subsection{Curvature}

\begin{equation}
    \kappa = \frac{3(\cos^2 + 1)}{R\sin\theta(3\cos^2\theta + 1)^{3/2}}
\end{equation}

\subsection{RVM}

\begin{equation}
    \tan{\psi} = \frac{\sin\alpha \sin\phase}{\sin\zeta\cos\alpha - \cos\zeta\sin\alpha\cos\phase}
\end{equation}

\begin{equation}
    \left.\frac{d\psi}{d\phase}\right|_{\phase=0} = \frac{\sin\alpha}{\sin\beta}
\end{equation}

\rule{\columnwidth}{1pt}

\section{Miscellaneous}

\begin{equation}
    f_c = \frac{3\gamma^3 c\kappa}{4\pi}
\end{equation}

\begin{equation}
    R = \begin{bmatrix}
        \cos\theta & -\sin\theta \\
        \sin\theta & \cos\theta
    \end{bmatrix}
\end{equation}

\begin{equation}
    \kappa = \frac{|\vec{v}\times\vec{a}|}{|\vec{v}|^3}
\end{equation}

\begin{equation}
    \kappa = \frac{|2(f^\prime(\theta))^2 + (f(\theta))^2 - f(\theta)f^{\prime\prime}(\theta)|}{\left[(f^\prime(\theta))^2 + (f(\theta))^2\right]^{3/2}}
\end{equation}


\end{document}
