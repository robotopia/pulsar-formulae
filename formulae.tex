\documentclass{article}

\usepackage{amsmath}
\usepackage{fullpage}
\usepackage[colorlinks]{hyperref}
\usepackage{bm}

\title{Formulas relating to pulsar geometry}
\author{Sam McSweeney}

\newcommand{\phase}{\varphi}
\newcommand{\deriv}[2]{\frac{d{#1}}{d{#2}}}
\newcommand{\derivn}[3]{\frac{d^{#3}{#1}}{d{#2}^{#3}}}
\newcommand{\pd}[2]{\frac{\partial {#1}}{\partial {#2}}}
\newcommand{\F}[1]{\bm{F}_{#1}}
\newcommand{\ph}{\phi_\ast}

\begin{document}

\maketitle

\tableofcontents

\section{Magnetic dipole equations}

In this section, unless otherwise specified, $\theta$ refers to the angle a given point makes with the magnetic axis.

\subsection{Aligned Rotator}

From symmetry, we can treat the aligned rotator as a 2D problem, with the magnetic axis aligned with the $y$-axis.

\subsubsection{Field lines}

The magnetic field, in coordinate-free notation:
\begin{equation}
    \vec{B} = \frac{1}{r^3}\left[3(\vec{\mu}\cdot\hat{r})\hat{r} - \vec{\mu}\right]
\end{equation}
Consider a single field line with maximum extent $R$.
It can be expressed in either Cartesian or polar coordinates.
The relations between the various coordinates are found in Table \ref{tbl:dipolar}.
\begin{table}[!ht]
    \centering
    \caption{Each coordinate in terms of the others}
    \label{tbl:dipolar}
    \begin{tabular}{c|cccc}
        & $r$ & $\theta$ & $x$ & $y$ \\[5pt]
        \hline
        $r =$      & --- & $R\sin^2\theta$ & $\sqrt[3]{Rx^2}$ & ? \\[8pt]
        $\theta =$ & $\sin^{-1}\sqrt{\dfrac{r}{R}}$ & --- & $\sin^{-1}\sqrt[3]{\dfrac{r}{R}}$ & ? \\[8pt]
        $x =$      & $\sqrt{\dfrac{r^3}{R}}$ & $R\sin^3\theta$ & --- & ? \\[8pt]
        $y =$      & $r\sqrt{1-\dfrac{r}{R}}$ & $R\sin^2\theta\cos\theta$ & $x^{2/3} \sqrt{R^{2/3} - x^{2/3}}$ & ---
    \end{tabular}
\end{table}

\subsubsection{Derivatives}

See Table \ref{tbl:dipolarderiv}.
We use the definition $d\ell = \sqrt{dx^2+dy^2}$.

\begin{table}[!p]
    \centering
    \caption{The derivatives of the expressions in Table \ref{tbl:dipolar}}
    \label{tbl:dipolarderiv}
    \begin{tabular}{c|cccc}
        & $r$ & $\theta$ & $x$ & $y$ \\[5pt]
        \hline
        $dr/d\theta =$ & $2\sqrt{r(R-r)}$ & $2R\sin\theta\cos\theta$ & $2R^{1/3}x^{1/3}\sqrt{R^{2/3}-x^{2/3}}$ & ? \\[8pt]
        $dr/dx =$      & $\dfrac23\sqrt{\dfrac{R}{r}}$ & $\dfrac{2}{3\sin\theta}$ & $\dfrac23\sqrt[3]{\dfrac{R}{x}}$ & \\[8pt]
        $dr/dy =$      & & & & \\[8pt]
        \hline
        $d\theta/dr =$ & & & & \\[8pt]
        $d\theta/dx =$ & & & & \\[8pt]
        $d\theta/dy =$ & & & & \\[8pt]
        \hline
        $dx/dr =$      & $\dfrac32\sqrt{\dfrac{r}{R}}$ & & & \\[8pt]
        $dx/d\theta =$ & & $3R\sin^2\theta\cos\theta$ & $3x^{2/3} \sqrt{R^{2/3} - x^{2/3}}$ & $3y$ \\[8pt]
        $dx/dy =$      & & $\dfrac{3\sin\theta\cos\theta}{3\cos^2\theta-1}$ & $\dfrac{3x^{1/3}\sqrt{R^{2/3} - x^{2/3}}}{2R^{2/3} - 3x^{2/3}}$ & \\[8pt]
        \hline
        $dy/dr =$      & $\dfrac{2R-3r}{2\sqrt{R(R-r)}}$ & & & \\[8pt]
        $dy/d\theta =$ & $\sqrt{\dfrac{r}{R}}\left(2R-3r\right)$ & $R\sin\theta(3\cos^2\theta - 1)$ & $x^{1/3}(2R^{2/3} - 3x^{2/3})$ & \\[8pt]
        $dy/dx =$      & & $\dfrac{3\cos^2\theta-1}{3\sin\theta\cos\theta}$ & $\dfrac{2R^{2/3} - 3x^{2/3}}{3x^{1/3}\sqrt{R^{2/3} - x^{2/3}}}$ & \\[8pt]
        \hline
        $d\ell/dr =$ & $\dfrac12 \sqrt{\dfrac{4R-3r}{R-r}}$ & & & \\[8pt]
        $d\ell/d\theta =$ & & $R\sin\theta \sqrt{3\cos^2\theta + 1}$ & &
    \end{tabular}
\end{table}

\subsubsection{Arc Length}
The arc length of a field line, as a function of $\theta$:
\begin{equation}
    \begin{aligned}
        \int\,d\ell = \int\deriv{\ell}{\theta}\,d\theta
            &= \int R\sin\theta \sqrt{3\cos^2\theta + 1} \, d\theta \\
            &= -\frac{R}{6}\left(3\cos\theta\sqrt{3\cos^2\theta + 1} + \sqrt{3}\sinh^{-1}(\sqrt{3}\cos\theta)\right) + C
    \end{aligned}
\end{equation}
In particular, the arc length as measured from the origin is
\begin{equation}
    \ell_0(\theta) = -\frac{R}{6}\left(3\cos\theta\sqrt{3\cos^2\theta + 1} + \sqrt{3}\sinh^{-1}(\sqrt{3}\cos\theta)\right) +
                      \frac{R\sqrt{3}\sinh^{-1}\sqrt{3}}{6} + R
\end{equation}
whose Taylor expansion about $\theta = 0$ gives
\begin{equation}
    \frac{\ell_0}{R} \approx \theta^2 - \frac{13}{48}\theta^4 + \frac{241}{5760}\theta^6 + O(\theta^8)
    \label{eqn:arclengthth}
\end{equation}
The arc length, $\ell$, as a function of $r$:
\begin{equation}
    \begin{aligned}
        \ell = \int\,d\ell = \int\deriv{\ell}{r}\,dr
            &= \int \frac12 \sqrt{\frac{4R-3r}{R-r}} \, dr \\
            &= \frac{R}{12}\left(\frac{6S}{3 - S^2} + \sqrt{3}\log\left(\frac{S - \sqrt{3}}{S + \sqrt{3}}\right)\right) + C
    \end{aligned}
\end{equation}
where
\begin{equation}
    S = \sqrt{\frac{4R-3r}{R-r}}
\end{equation}
In particular, the arc length as measured from the origin is
\begin{equation}
    \ell_0 = \int_0^r \deriv{s}{r^\prime} \,dr^\prime
           = \frac{R}{12}\left(\frac{6S}{3 - S^2} + \sqrt{3}\log\left(\frac{S - \sqrt{3}}{S + \sqrt{3}}\right)\right) +
             R - \frac{R\sqrt{3}}{12}\log\left(\frac{2-\sqrt{3}}{2+\sqrt{3}}\right)
\end{equation}
and the arc length as measured from the surface is
\begin{equation}
    \begin{aligned}
        \ell_\text{surface} &= \int_{r_p}^r \deriv{s}{r^\prime} \,dr^\prime \\
               &= \frac{R}{12}\left(\frac{6S}{3 - S^2} + \sqrt{3}\log\left(\frac{S - \sqrt{3}}{S + \sqrt{3}}\right)\right) -
                  \frac{R}{12}\left(\frac{6S_{r_p}}{3 - S_{r_p}^2} + \sqrt{3}\log\left(\frac{S_{r_p} -
                  \sqrt{3}}{S_{r_p} + \sqrt{3}}\right)\right)
     \end{aligned}
\end{equation}
where
\begin{equation}
    S_{r_p} = \sqrt{\frac{4R-3r_p}{R-r_p}}
\end{equation}
A Taylor series expansion of $\ell_0$ about $r=0$ gives
\begin{equation}
    \ell_0 \approx r + \frac{r^2}{16R} + \frac{5r^3}{128R^2} + \frac{113r^4}{4096R^3} + O(r^5)
\end{equation}
in agreement with Eq \eqref{eqn:arclengthth} above.

\subsubsection{Footpoints}

Let $r_p$ be the pulsar's radius, and $\theta_p$ by the angle the footpoint makes with the magnetic axis.
\begin{align}
    r_p &= R\sin^2\theta_p \\
    \theta_p &= \sin^{-1}\sqrt{\frac{r_p}{R}}
\end{align}
The $x$- and $y$-coordinates of the footpoint, $(x_p,y_p)$:
\begin{align}
    x_p &= r_p\sqrt{\frac{r_p}{R}} \\
    y_p &= r_p \sqrt{1 - \frac{r_p}{R}}
\end{align}
To normalise a footpoint with respect to the footpoint of the last open field lines ($R = r_L \equiv \frac{cP}{2\pi}$), we define
\begin{equation}
    \theta_L \equiv \sin^{-1}\sqrt{\frac{r_p}{r_L}}
             =      \sin^{-1}\sqrt{\frac{2\pi r_p}{cP}}
\end{equation}
and then
\begin{equation}
    s \equiv \frac{\theta_p}{\theta_L}
\end{equation}
To convert from a given $s$ to $R$:
\begin{equation}
    R = \frac{r_p}{\sin^2\left(s\sin^{-1}\sqrt{\frac{2\pi r_p}{cP}}\right)}
\end{equation}

\subsubsection{Curvature}

\paragraph{Without rotation effects:}
From Eq \eqref{eqn:curvature_polar},
\begin{equation}
    \begin{aligned}
        \kappa &= \frac{3(\cos^2 + 1)}{R\sin\theta(3\cos^2\theta + 1)^{3/2}} \\
        \rho = \frac{1}{\kappa}
            &= \frac{R\sin\theta(3\cos^2\theta + 1)^{3/2}}{3(\cos^2 + 1)}
    \end{aligned}
\end{equation}
Taylor expansions about $\theta = 0$:
\begin{equation}
    \begin{aligned}
        \kappa R &\approx \frac34 \theta^{-1} + \frac{19}{32} \theta + \frac{2347}{7680} \theta^3 + O(\theta^5) \\
        \frac{\rho}{R} &\approx \frac43 \theta - \frac{19}{18}\theta^3 + \frac{421}{1440}\theta^5 + O(\theta^7)
    \end{aligned}
\end{equation}
Substituting $R = r/\sin^2\theta$ yields
\begin{equation}
    \begin{aligned}
        \kappa r &\approx \frac34 \theta + \frac{11}{32} \theta^3 + \frac{361}{2560} \theta^5 + O(\theta^7) \\
        \frac{\rho}{r} &\approx \frac43 \theta^{-1} - \frac{11}{18}\theta + \frac{127}{4320}\theta^3 + O(\theta^5)
    \end{aligned}
\end{equation}

\paragraph{With rotation effects:}
Let $a \equiv R/r_L$, where $R$ is the distance of maximum extent for a given field lines, and $r_L = c/\Omega$ is the light cylinder radius.
Let $B \equiv \sin^2\theta = \frac{r}{R}$.
\begin{equation}
    \begin{aligned}
        \kappa &= 3\sqrt{
            \frac{-(a^4B^8 - 4a^4B^7 + 4a^4B^6 -
                   11a^2B^5 + 41a^2B^4 - 48a^2B^3 +
                   (16a^2+1)B^2 - 4B + 4)}{
                       Rr(27B^3 - 108B^2 + 144B - 64)}} \\
           &= 3\sqrt{
               \frac{-a^4B^6(B-2)(B+2) +
                   a^2B^2(11B^3 - 41B^2 + 48B - 16) -
                   (B-2)(B+2)}{
                       Rr(27B^3 - 108B^2 + 144B - 64)}}
    \end{aligned}
\end{equation}
A Taylor expansion about $\theta = 0$ gives
\begin{equation}
    \begin{aligned}
            \kappa &\approx \frac{3\sqrt{4r^2+r_L^2}}{4rr_L}\,\theta -
                            \frac{52r^2 - 11r_L^2}{32rr_L\sqrt{4r^2+r_L^2}}\,\theta^3 +
                            O(\theta^5) \\
            \rho &\approx \frac{4rr_L}{3\sqrt{4r^2+r_L^2}}\,\theta^{-1} +
                          \frac{rr_L(52r^2 - 11r_L^2)}{18(4r^2+r_L^2)^{3/2}}\,\theta +
                          O(\theta^3)
    \end{aligned}
\end{equation}

\subsubsection{Beam Opening Angles}
\label{sec:beamopeningangles}

\paragraph{Without rotation effects:}
The relationship between $\theta$ and $\Gamma$ (and its inverse) is given by\footnote{Gangadhara (2004)}
\begin{equation}
    \begin{aligned}
        \cos(2\theta) &= \frac13\left(\cos\Gamma\sqrt{8 + \cos^2\Gamma} - \sin^2\Gamma\right) \\
        \tan\Gamma &= \frac{3\sin\theta\cos\theta}{3\cos^2\theta - 1}
    \end{aligned}
\end{equation}
Taylor series expansions (about $\theta = 0$):
\begin{equation}
    \begin{aligned}
        \theta &\approx \frac23 \Gamma - \frac2{81} \Gamma^3 + O(\Gamma^7) \\
        \Gamma &\approx \frac32 \theta + \frac18 \theta^3 + \frac1{32} \theta^5 + O(\theta^7)
    \end{aligned}
\end{equation}

\paragraph{With rotation effects:}
\begin{equation}
    \cos\Gamma = \frac{\sqrt{r^2\cos^2\theta-r^2+r_L^2}(3\cos^2\theta-1)}{r_L\sqrt{3cos^2\theta+1}}
\end{equation}
Taylor series expansions (about $\theta = 0$):
\begin{equation}
    \begin{aligned}
        \cos\Gamma &\approx 1 - \frac{4r^2+9r_L^2}{8r_L^2}\,\theta^2 -
                            \frac{48r^4 - 280r^2r_L^2 - 9r_L^4}{384r_L^4}\,\theta^4 + O(\theta^6) \\
        \Gamma &\approx \frac{\sqrt{4r^2+9r_L^2}}{2r_L}\,\theta +
                        \frac{8r^4 - 26r^2r_L^2 + 9r_L^4}{24r_L^3\sqrt{4r^2+9r_L^2}}\,\theta^3 + O(\theta^5)
    \end{aligned}
\end{equation}

\subsection{Inclined Rotator}

In the following, $\alpha$ is the angle that the magnetic axis makes with the rotation axis, and $\zeta$ is the angle between the rotation axis and the line of sight.
The rotation axis is aligned with the $z$-axis, and unless otherwise specified, the formulas pertain to the time when the magnetic axis $\vec{\mu}$ is instantaneously in the $xz$-plane (and where $\vec{\mu}\cdot\hat{x}$ is positive).

\subsubsection{Magnetic Field in Cartesian Coordinates}

\begin{equation}
    \vec{B}(\vec{r}) =
    \frac{\mu\cos\alpha}{r^5}\begin{bmatrix}
        3xz \\
        3yz \\
        3z^2 - r^2
    \end{bmatrix} + 
    \frac{\mu\sin\alpha}{r^5}\begin{bmatrix}
        3x^2 - r^2 \\
        3xy \\
        3xz
    \end{bmatrix}
\end{equation}

\subsubsection{Rotating Vector Model (RVM)}

\begin{equation}
    \tan{\psi} = \frac{\sin\alpha \sin\phase}{\sin\zeta\cos\alpha - \cos\zeta\sin\alpha\cos\phase}
\end{equation}

\begin{equation}
    \left.\frac{d\psi}{d\phase}\right|_{\phase=0} = \frac{\sin\alpha}{\sin(\zeta-\alpha)}
\end{equation}

\subsubsection{Pulse Width}

The pulse width, $W$, is related to the beam opening angle, $\Gamma$, as follows (from the \emph{Handbook}):
\begin{equation}
    \cos\Gamma = \cos\alpha\cos\zeta + \sin\alpha\sin\zeta\cos\left(\frac{W}{2}\right)
    \tag{H3.27}
\end{equation}

\subsubsection{Beam Opening Angle}

In this section, $(r,\theta,\sigma)$ are the spherical coordinates in the magnetic frame.
The Taylor expansion about $\theta = 0$ for the full expression of the beam opening angle, $\Gamma$, including rotation effects, is
\begin{equation}
    \cos\Gamma \approx \frac{3\cos^2\alpha-1}{\sqrt{3\cos^2\alpha + 1}} +
                   \left(-\frac{r\sin\alpha(9\cos^2\alpha+1)\sin\sigma}{2r_L(3\cos^2\alpha + 1)} +
                          \frac{9\sin\alpha\cos\alpha(\cos^2\alpha+1)\cos\sigma}{(3\cos^2\alpha+1)^{3/2}}\right) \,\theta +
                   O(\theta^2)
\end{equation}
In the extreme cases, we have
\paragraph{$\alpha = 0$:}
\begin{equation}
    \cos\Gamma \approx 1 + \left(\frac{r^2}{2r_L^2} + \frac98\right)\,\theta^2 + O(\theta^4)
\end{equation}
\paragraph{$\alpha = \pi/2$:}
\begin{equation}
    \cos\Gamma \approx 1 - \frac{r\sin\sigma}{2r_L}\,\theta - \left(\frac{3r^2 + (r^2 + 36r_L^2)\cos^2\sigma}{8r_L^2}\right)\,\theta^2 + O(\theta^3)
\end{equation}

\section{Magnetic Deutsch/Arendt equations}

\begin{equation}
\begin{aligned}
    \vec{B}(\vec{r}) &= \mu \cos{\alpha} \F{1} \frac{1}{r^5} + \\
                   &\hspace{13pt} \mu \sin{\alpha} \F{2} \frac{\cos(\ph)}{r^3 r_L^2} +
                                  \mu \sin{\alpha} \F{3} \bigg(\frac{\cos(\ph)}{r^5} + \frac{\sin(\ph)}{r^4 r_L}\bigg) - \\
                   &\hspace{13pt} \mu \sin{\alpha} \F{4} \frac{\sin(\ph)}{r^3 r_L^2} -
                                  \mu \sin{\alpha} \F{5} \bigg(\frac{\sin(\ph)}{r^5} - \frac{\cos(\ph)}{r^4 r_L}\bigg)
\end{aligned}
\end{equation}
where
\begin{equation}
\begin{aligned}
    \F{1} &= 3xz\,\hat{x} + 3yz\,\hat{y} + (3z^2-r^2)\,\hat{z} \\
    \F{2} &= (r^2-x^2)\,\hat{x} - xy\,\hat{y} - xz\,\hat{z} \\
    \F{3} &= (3x^2-r^2)\,\hat{x} + 3xy\,\hat{y} + 3xz\,\hat{z} \\
    \F{4} &= -xy\,\hat{x} + (r^2-y^2)\,\hat{y} - yz\,\hat{z} \\
    \F{5} &= 3xy\,\hat{x} + (3y^2-r^2)\,\hat{y} + 3yz\,\hat{z}
\end{aligned}
\end{equation}
and
\begin{equation}
    \ph = \frac{r - r_p}{r_L}
\end{equation}

\section{Retardation}

Let $\hat{n}$ be the line of sight, a unit vector pointing towards the observer.
A particle at position $\vec{r}$ has a ``retardation time''
\begin{equation}
    \Delta\tau_\text{ret} = \frac{\vec{r}\cdot\hat{n}}{c}
\end{equation}
and a corresponding ``retardation phase''
\begin{equation}
    \Delta\phase_\text{ret} = -\frac{2\pi\Delta\tau_\text{ret}}{P} = -\frac{\vec{r}\cdot\hat{n}}{r_L}
\end{equation}
where $P$ is the pulsar's rotation period, $r_L$ is the light cylinder radius, and where it is understood that emission from a particle at $\vec{r}$ at some rotation phase $\phase$ would be observed at phase $\phase + \Delta\phase_\text{ret}$.

\section{Carousels and Sparks}
\label{sec:carousel}

Let $\theta_\text{cl}$ be the angular radius of the carousel, $N$ be the number of sparks, $\theta_\text{sp}$ be the angular radius of an individual spark, $P_4$ be the carousel rotation period, and $t$ be the current time.
We define spark $0$ to be the spark that lies in the $xz$-plane (of magnetic coordinates) at time $t=0$.
Spark 1 is the next spark counterclockwise (as viewed from above).
Rotation occurs in the same sense (i.e. spark 0 will head towards spark 1) for positive $P_4$; in the opposite sense otherwise.
Then the magnetic azimuth of the $n$th spark at time $t$ is
\begin{equation}
    \sigma_n(t) = 2\pi\left(\frac{n}{N} + \frac{t}{P_4}\right)
\end{equation}
An arbitrary point on the polar cap with magnetic azimuth $\sigma$ has the time-dependent spark phase
\begin{equation}
    \phase_\text{sp}(\sigma,t) = N\left(\sigma - \frac{2\pi t}{P_4}\right)
\end{equation}
and will be nearest in position to spark
\begin{equation}
    n_\text{nearest} = \left\lfloor N\left(\frac{\sigma}{2\pi} - \frac{t}{P_4}\right) + 0.5 \right\rfloor
\end{equation}
where $n_\text{nearest}$ is understood to be taken modulo $N$.
Let $D$ be the angular distance between a given footpoint $(\theta_\text{fp},\sigma_\text{fp})$ and the centre of the nearest spark at time $t$.
Then the spark profile is defined as
\begin{equation}
    H_\text{sp}(\theta_\text{fp},\sigma_\text{fp},t) \equiv e^{-9D^2/2\theta_\text{sp}}
\end{equation}
where the spark radius is identified with $3\sigma$ of a standard normal profile.

\section{Coordinate Systems}

In this section (unless otherwise specified), $\theta$ is the colatitude (measured from the $z$-axis) and $\phase$ is the longitude (measured from the $x$-axis).

\subsection{Spherical $\leftrightarrow$ Cartesian}

\subsubsection{Spherical $(r,\theta,\phase)$ $\rightarrow$ Cartesian $(x,y,z)$}
\begin{equation}
    \begin{aligned}
        x &= r \cos\phase\sin\theta \\
        y &= r \sin\phase\sin\theta \\
        z &= r \cos\theta
    \end{aligned}
\end{equation}
\subsubsection{Cartesian $(x,y,z)$ $\rightarrow$ Spherical $(r,\theta,\phase)$}
\begin{equation}
    \begin{aligned}
        r &= \sqrt{x^2+y^2+z^2} \\
        \theta &= \cos^{-1} \left(\frac{z}{r}\right) \\
        \phase &= \tan^{-1} \left(\frac{y}{x}\right)
    \end{aligned}
\end{equation}

\subsection{Euler rotations}

\subsubsection{2D Rotation}
\begin{equation}
    R(\theta) = \begin{bmatrix}
        \cos\theta & -\sin\theta \\
        \sin\theta & \cos\theta
    \end{bmatrix}
\end{equation}

\subsubsection{3D Rotation}
\begin{equation}
    R_x(\theta) = \begin{bmatrix}
        1 & 0 & 0 \\
        0 & \cos\theta & -\sin\theta \\
        0 & \sin\theta & \cos\theta
    \end{bmatrix}
\end{equation}
\begin{equation}
    R_y(\theta) = \begin{bmatrix}
        \cos\theta & 0 & \sin\theta \\
        0 & 1 & 0 \\
        -\sin\theta & 0 & \cos\theta
    \end{bmatrix}
\end{equation}
\begin{equation}
    R_z(\theta) = \begin{bmatrix}
        \cos\theta & -\sin\theta & 0 \\
        \sin\theta & \cos\theta & 0 \\
        0 & 0 & 1 \\
    \end{bmatrix}
\end{equation}

\subsection{Magnetic $\leftrightarrow$ Observer}

In the magnetic coordinate system, the $z$-axis is identified with the magnetic moment, and the $x$-axis is chosen to be in the plane containing both the rotation and magnetic axes.
The angle between the rotation and magnetic axes is, as usual, denoted by $\alpha$.
At rotation phase $\phase$, the relations between magnetic and observer coordinates are therefore defined as follows.

\subsubsection{Magnetic to Observer coordinates}
\begin{equation}
    R_z(\phase) R_y(\alpha) =
    \begin{bmatrix}
        \cos\phase\cos\alpha & -\sin\phase & \cos\phase\sin\alpha \\
        \sin\phase\cos\alpha &  \cos\phase & \sin\phase\sin\alpha \\
        -\sin\alpha          & 0           & \cos\alpha
    \end{bmatrix}
\end{equation}

\subsubsection{Observer to Magnetic coordinates}
\begin{equation}
    R_y(-\alpha) R_z(-\phase) =
    \begin{bmatrix}
        \cos\phase\cos\alpha & \sin\phase\cos\alpha & -\sin\alpha \\
        -\sin\phase          & \cos\phase           & 0           \\
        \cos\phase\sin\alpha & \sin\phase\sin\alpha &  \cos\alpha
    \end{bmatrix}
\end{equation}

\subsection{Spark $\leftrightarrow$ Magnetic coordinates}

(See \S\ref{sec:carousel} for the meaning of the notation in the following formulas.)

\subsubsection{Spark to Magnetic coordinates}
\begin{equation}
    \begin{bmatrix}
        \cos\phase_n(t)\cos\sigma & -\sin\phase_n(t) & \cos\phase_n(t)\sin\sigma \\
        \sin\phase_n(t)\cos\sigma &  \cos\phase_n(t) & \sin\phase_n(t)\sin\sigma \\
        -\sin\sigma          & 0           & \cos\sigma
    \end{bmatrix}
\end{equation}

\subsubsection{Magnetic to Spark coordinates}

[YET TO DO]

\section{Miscellaneous}

\subsection{Curvature}

\subsubsection{Curvature of arbitrarily accelerated particle}

\begin{equation}
    \kappa = \frac{|\vec{v}\times\vec{a}|}{|\vec{v}|^3}
\end{equation}

\subsubsection{Curvature of polar function $r = f(\theta)$}

\begin{equation}
    \kappa = \frac{|2(f^\prime(\theta))^2 + (f(\theta))^2 - f(\theta)f^{\prime\prime}(\theta)|}{\left[(f^\prime(\theta))^2 + (f(\theta))^2\right]^{3/2}}
    \label{eqn:curvature_polar}
\end{equation}

\subsection{Bessel functions}

\subsubsection{Expansions at $x=0$:}

\begin{equation}
    \lim_{x\rightarrow0} K_\nu(x)
        \approx \frac{\Gamma(\nu)}{2^{1-\nu}x^\nu}, \qquad (\nu > 0) \\
\end{equation}

\begin{align}
    \lim_{x\rightarrow0} K_{1/3}(x)
        &\approx 1.687625525626409\,x^{-1/3} \\
    \lim_{x\rightarrow0} K_{2/3}(x)
        &\approx 1.074764120767239\,x^{-2/3} \\
    \lim_{x\rightarrow0} K_{5/3}(x)
        &\approx 1.433018827689652\,x^{-5/3}
\end{align}

\subsubsection{Expansions at $x=\infty$:}

\begin{equation}
    \lim_{x\rightarrow\infty} K_\nu(x)
        \approx \sqrt{\frac{\pi}{2x}}e^{-x}
\end{equation}

\subsubsection{Upper integrals}

In practice, I often only evaluate the $K_\nu(x)$ family of functions (above) up to $x = 10$.
Therefore, in order to get accurate numerical upper integrals of the form $F_\nu(x) = \int_x^\infty K_\nu(x)\,dx$, it is useful to have some values of $F_\nu(10)$.
\begin{align}
    F_{1/3}(10) &= \int_{10}^\infty K_{1/3}(x)\,dx = 0.0000170985 \\
    F_{2/3}(10) &= \int_{10}^\infty K_{2/3}(x)\,dx = 0.0000173511 \\
    F_{5/3}(10) &= \int_{10}^\infty K_{5/3}(x)\,dx = 0.0000192238
\end{align}

\subsection{Stokes Parameters}

\subsubsection{From circularly polarised fields}

Let $E_L$ and $E_R$ be the amplitudes of the left and right circularly polarized electric field respectively, and let $\delta$ be the phase angle between them.
\begin{equation}
    \begin{aligned}
        I &= \left\langle E_L^2 \right\rangle + \left\langle E_R^2 \right\rangle \\
        Q &= 2\left\langle E_L E_R \cos\delta \right\rangle \\
        U &= 2\left\langle E_L E_R \sin\delta \right\rangle \\
        V &= \left\langle E_R^2 \right\rangle - \left\langle E_L^2 \right\rangle
    \end{aligned}
\end{equation}
In terms of \emph{intensity}, $I_\pm \equiv \frac{d^2I_\pm}{d\omega\,d\Omega} = E_{R/L}^2$ (see equation \ref{eqn:JP14_25} below),
\begin{equation}
    \begin{aligned}
        I &= I_+ + I_- \\
        V &= I_+ - I_- \\
        L = \sqrt{Q^2 + U^2} &= 2\sqrt{I_+ I_-}
    \end{aligned}
\end{equation}

\subsection{Spherical distance}

Let $(\theta_1,\phase_1)$ and $(\theta_2,\phase_2)$ be two points on the unit sphere, where $\theta$ denotes colatitude and $\phase$ denotes longitude.
The great-circle angle between the two points is\footnote{According to Wikipedia (``Great-circle distance'', accessed 2018-06-23), this formula becomes inaccurate on computer systems with low floating-point precision if the distance is small. However, for 64-bit floating point numbers, it ``does not have serious rounding errors for distances larger than a few meters on the surface of the Earth.''}
\begin{equation}
    \delta = \cos^{-1}(\cos\theta_1 \cos\theta_2 + \sin\theta_1 \sin\theta_2 \cos(\phase_1 - \phase_2))
\end{equation}

\section{Jackson's Classical Electrodynamics}

A convenient reparameterisation:
\begin{equation}
    \xi = \frac{\omega\rho}{3c}\left(\frac{1}{\gamma^2} + \theta^2\right)^{3/2}
    \tag{J14.76}
\end{equation}
Spectral intensity as a function of frequency and direction:
\begin{equation}
    \frac{d^2I}{d\omega\,d\Omega} =
        \frac{e^2}{3\pi^2c}
        \left( \frac{\omega\rho}{c} \right)^2
        \left( \frac{1}{\gamma^2} + \theta^2 \right)^2
        \left[ K_{2/3}^2(\xi) + \frac{\theta^2}{(1/\gamma^2) + \theta^2} K_{1/3}^2(\xi) \right]
    \tag{J14.79}
\end{equation}
As above, but made explicit for positive and negative helicity:
\begin{equation}
    \frac{d^2I_\pm}{d\omega\,d\Omega} =
        \frac{e^2}{6\pi^2c}
        \left( \frac{\omega\rho}{c} \right)^2
        \left( \frac{1}{\gamma^2} + \theta^2 \right)^2
        \left| K_{2/3}(\xi) \pm \frac{\theta}{\left(\frac{1}{\gamma^2} + \theta^2\right)^{1/2}} K_{1/3}(\xi) \right|^2
    \tag{J Problem 14.25}
    \label{eqn:JP14_25}
\end{equation}
Critical frequency:
\begin{equation}
    \begin{aligned}
        f_c &= \frac{3\gamma^3 c\kappa}{4\pi}
             = \frac{3\gamma^3 c}{4\pi\rho} \\
        \omega_c &= \frac{3\gamma^3 c\kappa}{2}
                  = \frac{3\gamma^3 c}{2\rho}
    \end{aligned}
    \tag{J14.81}
\end{equation}


\end{document}
