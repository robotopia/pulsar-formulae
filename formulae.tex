\documentclass[twocolumn]{article}

\usepackage{amsmath}
\usepackage{fullpage}
\usepackage{hyperref}

\newcommand{\phase}{\varphi}
\newcommand{\deriv}[2]{\frac{d{#1}}{d{#2}}}
\newcommand{\derivn}[3]{\frac{d^{#3}{#1}}{d{#2}^{#3}}}
\newcommand{\pd}[2]{\frac{\partial {#1}}{\partial {#2}}}

\begin{document}

\tableofcontents

\section{Magnetic dipole equations}

\subsection{Field lines}

The magnetic field, in coordinate-free notation:
\begin{equation}
    \vec{B} = \frac{1}{r^3}\left[3(\vec{\mu}\cdot\hat{r})\hat{r} - \vec{\mu}\right]
\end{equation}
A single field line with maximum extent $R$, in polar coordinates:
\begin{equation}
    r = R\sin^2\theta
\end{equation}
The $x$- and $y$-coordinates of the above:
\begin{align}
    x &= R\sin^3\theta \\
    y &= R\sin^2\theta\cos\theta
\end{align}
And finally, the field line in Cartesian coordinates, where the $y$ axis is identified with the magnetic axis:
\begin{equation}
    y = x^{2/3} \sqrt{R^{2/3} - x^{2/3}}
\end{equation}

\subsection{Derivatives}

\begin{equation}
    \begin{aligned}
        \deriv{x}{\theta} &= 3R\sin^2\theta\cos\theta \\
          &= 3x^{2/3} \sqrt{R^{2/3} - x^{2/3}} \\
          &= 3y
    \end{aligned}
\end{equation}

\begin{equation}
    \begin{aligned}
        \deriv{y}{\theta} &= R\sin\theta(3\cos^2\theta - 1) \\
                          &= x^{1/3}(2R^{2/3} - 3x^{2/3})
    \end{aligned}
\end{equation}
The tangent of a field line:
\begin{equation}
    \begin{aligned}
        \deriv{y}{x} &= \frac{3\cos^2\theta-1}{3\sin\theta\cos\theta} \\
                     &= \frac{2R^{2/3} - 3x^{2/3}}{3x^{1/3}\sqrt{R^{2/3} - x^{2/3}}}
    \end{aligned}
\end{equation}
The beam opening angle, $\Gamma$, is
\begin{equation}
    \tan\Gamma = \deriv{x}{y} = \frac{3\sin\theta\cos\theta}{3\cos^2\theta-1}
\end{equation}
Solving the above for $\theta$, we have (for $-\pi \le \Gamma \le \pi$)
\begin{equation}
    \cos(2\theta) = \frac13 \left(\cos\Gamma\sqrt{8+\cos^2\Gamma} - \sin^2\Gamma\right)
\end{equation}
The arc length of a field line, as a function of $\theta$:
\begin{equation}
    \begin{aligned}
        \deriv{s}{\theta} &= \sqrt{\left(\deriv{x}{\theta}\right)^2 + \left(\deriv{y}{\theta}\right)^2} \\
                          &= R\sin\theta \sqrt{3\cos^2\theta + 1}
    \end{aligned}
\end{equation}

\subsection{Footpoints}

Let $r_p$ be the pulsar's radius, and $\theta_p$ by the angle the footpoint makes with the magnetic axis.
\begin{align}
    r_p &= R\sin^2\theta_p \\
    \theta_p &= \sin^{-1}\sqrt{\frac{r_p}{R}}
\end{align}
The $x$- and $y$-coordinates of the footpoint, $(x_p,y_p)$:
\begin{align}
    x_p &= r_p\sqrt{\frac{r_p}{R}} \\
    y_p &= r_p \sqrt{1 - \frac{r_p}{R}}
\end{align}

\subsection{Curvature}

\begin{equation}
    \kappa = \frac{3(\cos^2 + 1)}{R\sin\theta(3\cos^2\theta + 1)^{3/2}}
\end{equation}

\subsection{RVM}

\begin{equation}
    \tan{\psi} = \frac{\sin\alpha \sin\phase}{\sin\zeta\cos\alpha - \cos\zeta\sin\alpha\cos\phase}
\end{equation}

\begin{equation}
    \left.\frac{d\psi}{d\phase}\right|_{\phase=0} = \frac{\sin\alpha}{\sin\beta}
\end{equation}

\rule{\columnwidth}{1pt}

\section{Retardation}

Let $\hat{n}$ be the line of sight, a unit vector pointing towards the observer.
A particle at position $\vec{r}$ has a ``retardation time''
\begin{equation}
    \Delta\tau_\text{ret} = \frac{\vec{r}\cdot\hat{n}}{c}
\end{equation}
and a corresponding ``retardation phase''
\begin{equation}
    \Delta\phase_\text{ret} = \frac{2\pi\Delta\tau_\text{ret}}{P},
\end{equation}
where $P$ is the pulsar's rotation period, and where it is understood that emission from a particle at $\vec{r}$ at some rotation phase $\phase$ would be observed at phase $\phase - \Delta\phase_\text{ret}$.

\section{Jackson's Classical Electrodynamics}

\begin{equation}
    f_c = \frac{3\gamma^3 c\kappa}{4\pi}
    \tag{J14.81}
\end{equation}

\section{Coordinate Systems}

In this section (unless otherwise specified), $\theta$ is the colatitude (measured from the $z$-axis) and $\phase$ is the longitude (measured from the $x$-axis).

\subsection{Spherical $\leftrightarrow$ Cartesian}

\subsubsection*{Spherical $(r,\theta,\phase)$ $\rightarrow$ Cartesian $(x,y,z)$}
\begin{equation}
    \begin{aligned}
        x &= r \cos\theta\sin\phase \\
        y &= r \sin\theta\sin\phase \\
        z &= r \cos\phase
    \end{aligned}
\end{equation}
\subsubsection*{Cartesian $(x,y,z)$ $\rightarrow$ Spherical $(r,\theta,\phase)$}
\begin{equation}
    \begin{aligned}
        r &= \sqrt{x^2+y^2+z^2} \\
        \theta &= \cos^{-1} \left(\frac{z}{r}\right) \\
        \phase &= \tan^{-1} \left(\frac{y}{x}\right)
    \end{aligned}
\end{equation}

\subsection{Euler rotations}

\subsubsection*{2D Rotation}
\begin{equation}
    R(\theta) = \begin{bmatrix}
        \cos\theta & -\sin\theta \\
        \sin\theta & \cos\theta
    \end{bmatrix}
\end{equation}

\subsubsection*{3D Rotation}
\begin{equation}
    R_x(\theta) = \begin{bmatrix}
        1 & 0 & 0 \\
        0 & \cos\theta & -\sin\theta \\
        0 & \sin\theta & \cos\theta
    \end{bmatrix}
\end{equation}
\begin{equation}
    R_y(\theta) = \begin{bmatrix}
        \cos\theta & 0 & \sin\theta \\
        0 & 1 & 0 \\
        -\sin\theta & 0 & \cos\theta
    \end{bmatrix}
\end{equation}
\begin{equation}
    R_z(\theta) = \begin{bmatrix}
        \cos\theta & -\sin\theta & 0 \\
        \sin\theta & \cos\theta & 0 \\
        0 & 0 & 1 \\
    \end{bmatrix}
\end{equation}

\subsection{Magnetic $\leftrightarrow$ Observer}

In the magnetic coordinate system, the $z$-axis is identified with the magnetic moment, and the $x$-axis is chosen to be in the plane containing both the rotation and magnetic axes.
The angle between the rotation and magnetic axes is, as usual, denoted by $\alpha$.
At rotation phase $\phase$, the relations between magnetic and observer coordinates are therefore defined as follows.

\subsubsection*{Magnetic to Observer coordinates}
\begin{equation}
    R_z(\phase) R_y(\alpha) =
    \begin{bmatrix}
        \cos\phase\cos\alpha & -\sin\phase & \cos\phase\sin\alpha \\
        \sin\phase\cos\alpha &  \cos\phase & \sin\phase\sin\alpha \\
        -\sin\alpha          & 0           & \cos\alpha
    \end{bmatrix}
\end{equation}

\subsubsection*{Observer to Magnetic coordinates}
\begin{equation}
    R_y(-\alpha) R_z(-\phase) =
    \begin{bmatrix}
        \cos\phase\cos\alpha & \sin\phase\cos\alpha & -\sin\alpha \\
        -\sin\phase          & \cos\phase           & 0           \\
        \cos\phase\sin\alpha & \sin\phase\sin\alpha &  \cos\alpha
    \end{bmatrix}
\end{equation}

\subsection{Spark $\leftrightarrow$ Magnetic coordinates}
Let $\sigma$ be the angular radius of the carousel, $N$ be the number of sparks, $P_4$ be the carousel rotation period, and $t$ be the current time.
We define spark $0$ to be the spark that lies in the $xz$-plane (of magnetic coordinates) at time $t=0$.
Spark 1 is the next spark counterclockwise (as viewed from above).
Rotation occurs in the same sense (i.e. spark 0 will head towards spark 1) for positive $P_4$; in the opposite sense otherwise.
Then the magnetic azimuth of the $n$th spark at time $t$ is
\begin{equation}
    \phase_n(t) = 2\pi\left(\frac{n}{N} + \frac{t}{P_4}\right)
\end{equation}
The transformation matrices are

\subsubsection*{Spark to Magnetic coordinates}
\begin{equation}
    \begin{bmatrix}
        \cos\phase_n(t)\cos\sigma & -\sin\phase_n(t) & \cos\phase_n(t)\sin\sigma \\
        \sin\phase_n(t)\cos\sigma &  \cos\phase_n(t) & \sin\phase_n(t)\sin\sigma \\
        -\sin\sigma          & 0           & \cos\sigma
    \end{bmatrix}
\end{equation}

\subsubsection*{Magnetic to Spark coordinates}

\section{Miscellaneous}

\begin{equation}
    \kappa = \frac{|\vec{v}\times\vec{a}|}{|\vec{v}|^3}
\end{equation}

\begin{equation}
    \kappa = \frac{|2(f^\prime(\theta))^2 + (f(\theta))^2 - f(\theta)f^{\prime\prime}(\theta)|}{\left[(f^\prime(\theta))^2 + (f(\theta))^2\right]^{3/2}}
\end{equation}


\end{document}
